Consider a region \(M \subset \R^3\) bounded by a smooth \emph{perfect conductor} \(\de M\), having outer unit normal \(\nu \colon \de M \to \R^3\). The static Maxwell equations in vacuum are
\begin{align*}
    \curl \mathbf{E} &= 0 && \text{(Faraday)}\\
    \dive \mathbf{E} &= \frac{\rho}{\eps_0} & &\text{(Gauss)}\\
    \dive \mathbf{B} &= 0 && \text{(no magnetic charges)}\\
    \curl \mathbf{B} &= \mu_0\mathbf{j} && \text{(Ampère)}
\end{align*}
with boundary conditions
\begin{equation}\label{eq: BC vectors}
    \mathbf{E} \times \nu = 0, \qquad \mathbf{B} \cdot \nu = 0  \qquad \text{on }\de M.
\end{equation}
where the electric field \(\mathbf{E} \colon M \to \R^3\) and the magnetic field \(\mathbf{B} \colon M \to \R^3\) are the unknowns, while the charge density \(\rho \in C^\infty_c(M)\), the current density \(\mathbf{j} \in C^\infty_c(M;\R^3)\), and the electric and magnetic permittivity constants in vacuum \(\eps_0,\mu_0 >0\) (that from now on we assume to be equal to 1, after having chosen a suitable unit system) are the data.

If we consider the differential forms
\begin{align*}
        E &\coloneq E_1 \dif x^1 + E_2 \dif x^2 + E_3 \dif x^3,\\
        B &\coloneq B_1 \dif x^2 \wedge \dif x^3 - B_2 \dif x^2 \wedge \dif x^3 + B_3 \dif x^1 \wedge \dif x^2\\
        J &\coloneq j_1 \dif x^1 + j_2 \dif x^2 + j_3 \dif x^3 
\end{align*}
a simple computation shows that the Maxwell equations become
\begin{align}
    \dif E = 0 && \delta E = \rho \\
    \dif B = 0 && \delta B = J
\end{align}
with boundary conditions
\begin{equation}\label{eq: BC forms}
    E|_{\de M} = 0, \qquad B|_{\de M} = 0.
\end{equation}
In fact, if we consider the simple case where \(\de M \approx \{x^3=0\}\) and \(M \approx \{x^3<0\}\), so that \(\nu \approx e_3\), then the boundary conditions \eqref{eq: BC vectors} become
\[
    E_1=E_2= 0, \qquad B_3=0.
\]
On the other hand,
\[
    E|_{\de M} = E_1 \dif x^2 + E_2 \dif x^2, \qquad B|_{\de M} = B_3 \dif x^1 \wedge \dif x^2,
\]
so the two sets of boundary conditions coincide.

Now we apply the results of the previous section to analyze topologically the system of Maxwell equations.
By Theorem~\ref{thm: Hodge with BC}, since \(E\) and \(B\) are closed
\[
    E = \alpha + \dif \phi, \qquad B = \beta + \dif A
\]
where \(\alpha \in \Hcal^1_D, \beta \in \Hcal^2_D\) and \(\phi \in \Omega^0_D, A \in \Omega^1_D\). Actually, we can assume that \(\phi\) and \(A\) are co-closed. In fact, \(\phi\) is a 0-form, so it's trivially co-closed. To fix \(A\), do a gauge transform \(A'=A+\dif \psi\), for some \(\psi \in \Omega^0_D\), so that \(\dif A'=\dif A\) and \(A' \in \Omega^1_D\). Then,
\[
    \delta A' = \delta A + \triangle \psi,
\]
so we need to solve the equation 
\[
    \triangle \psi = - \delta A
\]
with Dirichlet boundary conditions. Since \(\delta A \in (\Hcal^0_D)^\perp\), this problem have a unique solution, and we can replace \(A\) with the co-closed form \(A'\).

Now, using the inhomogeneous Maxwell equations and the co-closedness of \(\phi\) and \(A\),
\[
    \triangle\phi =  \rho, \qquad \triangle A =  J.
\]
Thus, solution to the Maxwell equations exist if and only if \(\rho \in (\Hcal^0_D)^\perp\) and \(J \in (\Hcal^1_D)^\perp\), i.e.,
\[
    \int_M \rho \vp = 0 \qquad \qquad \forall \vp \in \Hcal^0_D
\]
and
\[
    \int_M \alpha \wedge \star J = 0 \qquad \qquad \forall \alpha \in \Hcal^1_D.
\]

In our case, since \(M \subset \R^3\) is a bounded domain, by the usual elliptic theory the only solution of the boundary value problem
\begin{align*}
    \Delta u &= 0 \qquad \text{ on }M\\
    u &= 0  \qquad \text{ on } \de M
\end{align*}
is \(u=0\), so
\[
    \Hcal^0_D(M) = 0.
\]
This means that there are no conditions on the charge density.

Moreover, by Theorem~\ref{thm: Hodge - de Rham with Dirichlet BC}, \(\alpha\) and \(\beta\) determine the cohomology class relative to the boundary of the electric and the magnetic field, respectively.

Therefore, since it is the reason of the existence of non trivial harmonic forms, the presence of topology forces
\begin{itemize}
    \item compatibility conditions on the data
    \item non uniqueness of the solution to the boundary value problems.
\end{itemize}

In the rest of this section, I will examine a few examples. To make the interpretation easier, we will use the following deep theorem that relates the cohomology relative to the boundary \(H^k_{dR}(M,\de M)\) with the singular homology with real coefficients \(H_k(M;\R)\) (see \cite[Theorem~3.43]{HatcherAlgebraic_2000}).
\begin{thm}[Poincaré-Lefschetz duality]\label{thm: Poincaré-Lefscetz duality}
    Let \(M\) be a compact\footnote{In particular, it is a manifold of finite type, so cohomology and homology groups (with real coefficients) are finite dimensional vector spaces.} \(n\)-manifold with boundary \(\de M\). Then the map
    \begin{align*}
        H^k_{dR}(M,\de M) \times H^{n-k}_{dR}(M) &\to \R \\
        ([\omega], [\eta]) \mapsto \int_M \omega\wedge \eta
    \end{align*}
    is a well defined non degenerate pairing. In other words, by de Rham theorem\footnote{The de Rham theorem states that the de Rham cohomology is equivalent to the singular cohomology with real coefficients, i.e.,
    \[
        H^p_{dR}(M) \equiv H^p(M;\R) = H_p(M;\R)^*.
    \]
    See \cite[Chapter~18]{LeeSmooth_2013}.}
    \[
        H^k_{dR}(M,\de M) \equiv H_{n-k}(M;\R).
    \]
    canonically.
\end{thm}

\begin{ex}[Solid torus]
    Consider a solid torus 
    \[
        M = \{ (\cos \theta x, \sin \theta x, z) \in \R^3 \colon (x-2)^2+z^2 \le 1, \ \theta \in [0,2\pi]\} \cong \D^2 \times \Sp^1.
    \]
    Then \(M\) is homotopically equivalent to \(\Sp^1\), so \(H_\bullet(M) = H_\bullet(\Sp^1) = \R[t]/(t^2)\), and by Theorem~\ref{thm: Poincaré-Lefscetz duality}
    \begin{align*}
        H^0_{dR}(M,\de M) = 0,  & &H^1_{dR}(M, \de M)=0, \\
        H^2_{dR}(M,\de M) = \R, & &H^3_{dR}(M,\de M) = \R.
    \end{align*}
    The non triviality of \(H^2_{dR}(M,\de M)\) is caused by the presence of the non trivial loop \(\{0\} \times \Sp^1\), the ‘‘soul'' of the solid torus. Let us denote by \(\beta \in \Hcal^2_D(M)\) the harmonic representative of the non trivial class associated to it. 
    
    The electric field is always determined by \(E = \dif \phi\), where the scalar potential \(\phi \in \Omega^0_D(M)\) is the unique solution of the Poisson equation
    \[
        \triangle \phi = \rho,
    \]
    which always exists because \(\Hcal^0_D(M) \cong H^0_{dR}(M,\de M) = 0\).
    
    The magnetic field is \(B = \beta_B + \dif A\), where \(\beta_B \in \Hcal^2_D(M)\) is the harmonic representative of the class of \(B\) and \(A \in \Omega^1_D\) is the solution of
    \[
        \triangle A = J,
    \]
    which always exists because \(\Hcal^1_D(M) \cong H^1_{dR}(M,\de M) = 0\).
\end{ex}

\begin{ex}[Ball with spherical hole]
    Let 
    \[
        M= \{ x \in \R^3 \colon 1\le |x| \le 2\} \cong \Sp^2 \times [0,1].
    \]
    Then \(M\) is homotopically equivalent to \(\Sp^2\), so \(H_\bullet(\Sp^2;\R) = (\R[t]/(t^2), \deg(t)=2)\), and by Theorem~\ref{thm: Poincaré-Lefscetz duality}
    \begin{align*}
        H^0_{dR}(M,\de M) = 0,  & &H^1_{dR}(M, \de M)=\R, \\
        H^2_{dR}(M,\de M) = 0, & &H^3_{dR}(M,\de M) = \R.
    \end{align*}
    The non triviality of \(H^1_{dR}(M,\de M)\) is caused by the presence of the non trivial ‘‘middle sphere'' \(\Sp^2 \times \{1/2\}\).  

    The electric field is \(E = \alpha_E + \dif \phi\), where \(\alpha_E \in \Hcal^1_D(M)\) is the harmonic representative of the class of \(E\) and \(\phi \in \Omega^0_D\) is the solution of
    \[
        \triangle \phi = \rho,
    \]
    which always exists because \(\Hcal^0_D(M) \cong H^0_{dR}(M,\de M) = 0\).

    The magnetic field is \(B = \dif A\), where \(A \in \Omega^1_D\) is the solution of
    \[
        \triangle A = J,
    \]
    which exists if and only if the current density satisfies the compatibility condition
    \[
        \int_M \alpha \wedge \star J = 0,
    \]
    where \(\alpha \in \Hcal^1_D(M)\) is the harmonic representative of any non trivial class of \(H^1_{dR}(M,\de M)\), for instance the one associated to the ‘‘middle sphere''.
\end{ex}

\begin{ex}[Ball with toroidal hole]
    Let \(M = \B^3 \setminus N \subset \R^3\), where
    \[
    \begin{split}
        N &\coloneq \{ (\cos \theta x, \sin \theta x, z) \in \R^3 \colon (x-1/2)^2+z^2 < 1/16, \ \theta \in [0,2\pi]\} \\
        &\cong \D^2 \times \Sp^1.
    \end{split}
    \]
    A simple computation using the Mayer-Vietoris theorem (\(\B^3 = M \cup V\), with \(V \sim N\) a small neighborhood of \(N\)) shows that
    \begin{align*}
        H_0(M) = \R,  & &H_1(M)=\R, \\
        H_2(M) = \R, & &H_3(M) = 0.
    \end{align*}
    By Theorem~\ref{thm: Poincaré-Lefscetz duality},
    \begin{align*}
        H^0_{dR}(M,\de M) = 0,  & &H^1_{dR}(M, \de M)=\R, \\
        H^2_{dR}(M,\de M) = \R, & &H^3_{dR}(M,\de M) = \R.
    \end{align*}
    The non triviality of \(H^2_{dR}\) is caused by the presence of the loop
    \[
        \Gamma \coloneq \{(x,0,z) \in \R^3 \colon (x-1/2)^2 + z^2 = 1/8\} \cong \Sp^1,
    \]
    while the non triviality of \(H^1_{dR}(M)\) is caused by the presence of the ‘‘middle torus''
    \[
        T \coloneq \{ (\cos \theta x, \sin \theta x, z) \in \R^3 \colon (x,z) \in \Gamma, \ \theta \in [0,2\pi]\} \cong \Sp^1 \times \Sp^1.
    \]

    The electric field is \(E = \alpha_E + \dif \phi\), where where \(\alpha_E \in \Hcal^1_D(M)\) is the harmonic representative of the class of \(E\) and \(\phi \in \Omega^0_D\) is the solution of
    \[
        \triangle \phi = \rho,
    \]
    which always exists because \(\Hcal^0_D(M) \cong H^0_{dR}(M,\de M) = 0\).

    The magnetic field is \(B = \beta_B + \dif A\), where \(\beta_B \in \Hcal^2_D(M)\) is the harmonic representative of the class of \(B\) and \(A \in \Omega^1_D\) is the solution of
    \[
        \triangle A = J,
    \]
    which exists if and only if the current density satisfies the compatibility condition
    \[
        \int_M \alpha \wedge \star J = 0,
    \]
    where \(\alpha \in \Hcal^1_D(M)\) is the harmonic representative of any non trivial class of \(H^1_{dR}(M,\de M)\), for instance the one associated to the ‘‘middle torus'' \(T\).
\end{ex}