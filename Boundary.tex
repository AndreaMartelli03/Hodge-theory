In this section, we see the analogous of the Hodge theory, but for compact manifold with boundary. For a detailed exposition, see \cite{SchwarzHodge_1995}. 

Let \((M,g)\) be a compact connected oriented Riemannian \(n\)-manifold with non empty boundary \(\de M\). A possible definition of co-boundary as the \(L^2\)-adjoint of \(\dif\) would not give rise to a self-adjoint Laplacian, because boundary terms would appear. For this reason, we give the definition of co-boundary using Proposition~\ref{prop: co-boundary}:
\[
    \delta \coloneq (-1)^{kn+n+1}\star \dif \star.
\]

\begin{lemma}[Green]\label{lem: by parts with boundary}
    For any \(\omega \in \Omega^{k}(M)\) and \(\eta \in \Omega^{k-1}(M)\),
    \[
        \int_M g(\omega, \dif \eta) = \int_M g(\delta \omega, \eta) + \int_{\de M}(\eta \wedge \star \omega)|_{\de M}.
    \]
\end{lemma}
\begin{proof}
    Compute as in the proof of Proposition~\ref{prop: co-boundary}, but without canceling the boundary term given by Stokes theorem.
\end{proof}

\begin{defn}
    We say that \(\omega \in \Omega^k(M)\) satisfies 
    \begin{itemize}
        \item the \emph{(homogeneous) Dirichlet boundary condition} if 
        \[
            \omega|_{\de M} =0;
        \] 
        \item the \emph{(homogeneous) Neumann boundary condition} if
        \[
            (\star \omega)|_{\de M} = 0.
        \]
    \end{itemize}
    
    We denote
    \begin{align*}
        \Omega^k_D(M) &\coloneq \{ \omega \in \Omega^k(M) \colon \omega|_{\de M}=0\},\\
        \Omega^k_N(M) &\coloneq \{\omega \in \Omega^k(M) \colon (\star \omega)|_{\de M}=0\}
    \end{align*}
\end{defn}

\begin{lemma}\label{lemma: dOmega_D subset Omega_D}
    \(\dif \Omega^{k-1}_D(M) \subset \Omega^k_D(M)\) and \(\delta \Omega^{k+1}_N(M) \subset \Omega^k_N(M)\).
\end{lemma}
\begin{proof}
    Let's prove that \(\dif \Omega^{k-1}_D(M) \subset \Omega^k_D(M)\). For any \(\omega \in \Omega^{k-1}_D\), \((\dif \eta)|_{\de M} = \dif (\eta|_{\de M}) = 0\).

    To prove \(\delta \Omega^{k+1}_N(M) \subset \Omega^k_N(M)\), just observe that \(\omega \in \Omega^{k+1}_N(M)\) if and only if \(\star \omega \in \Omega^{n-k-1}_D(M)\), so \(0=(\dif \star\omega)_{\de M} = \pm (\star\delta\omega)|_{\de M}\).
\end{proof}

Denote also 
\[
    \Hcal^k(M) \coloneq \ker \dif|_{\Omega^k(M)} \cap \ker \delta|_{\Omega^k(M)},
\]
and 
\[ 
    \Hcal^k_D(M) \coloneq \Hcal^k(M) \cap \Omega^k_D(M), \qquad 
    \Hcal^k_N(M) \coloneq \Hcal^k(M) \cap \Omega^k_N(M).
\]
Clearly, the Hodge star gives an isomorphism between \(\Hcal^k_D(M)\) and \(\Hcal^{n-k}_N(M)\).


We can define the Hodge Laplacian as before by 
\[
    \triangle \coloneq \dif \delta + \delta \dif.
\] 
However, we did not defined an harmonic form \(\omega\) by the formula \(\triangle \omega = 0\), but rather by requiring that \(\omega\) is both closed and co-closed, in virtue of Lemma~\ref{lemma: harmonic iff closed and co-closed}. The problem is that \(\triangle\) is not a self-adjoint operator anymore, so the requirement \(\triangle \omega = 0\) is weaker.

However, by Lemma~\ref{lem: by parts with boundary}, \(\triangle\) is self-adjoint on the subspace
\[
    \mathcal{X}^k \coloneq \{ \omega \in \Omega^k(M) \colon \omega|_{\de M} = 0, \ (\delta \omega)|_{\de M} = 0\}.
\]
Therefore, the Fredholm alternative yields the following.
\begin{prop}\label{prop: fredholm}
    For any \(\theta \in \Omega^k(M)\), the problem 
    \begin{equation}\label{eq: BV pbm}
        \triangle \omega = \theta, \qquad \omega|_{\de M} = 0, \ (\delta \omega)|_{\de M} = 0        
    \end{equation}
    has a solution if and only if \(\theta\) is \(L^2\)-orthogonal to
    \[
        \Hcal^k_\triangle(M) \coloneq \{ \eta \in \Omega^k(M) \colon \triangle \eta = 0, \eta|_{\de M}=0, (\delta\eta)|_{\de M}=0 \}.
    \]
    In that case, the solution is unique.
\end{prop}
\begin{rmk}
    Observe that \(\Hcal^k_D(M) \subset \Hcal^k_\triangle(M)\). Thus a necessary condition for the existence of a solution of the problem \eqref{eq: BV pbm} is that \(\theta \in \Hcal^k_D(M)^\perp\).
\end{rmk}

It holds the analogous result of Theorem~\ref{thm: Hodge} (see \cite[Theorem~2.2.2, Theorem~2.2.7, Theorem~2.4.2, Theorem~2.4.8, Corollary~2.4.9]{SchwarzHodge_1995}).

\begin{thm}[Hodge-Morrey-Friedrichs]\label{thm: Hodge with BC}
    The spaces \(\Hcal^k_D\) and \(\Hcal^k_N\) are finite dimensional. Moreover, there hold the following \(L^2\)-orthogonal decompositions:
    \begin{align*}
        \Omega^k(M) &= \Hcal^k(M) \oplus \dif \Omega^{k-1}_D(M) \oplus \delta \Omega^{k+1}_N(M) &&\text{(Hodge-Morrey)}\\
        \Hcal^k(M) &= \Hcal^k_D(M) \oplus (\Hcal^k(M) \cap \delta\Omega^{k+1}(M)) &&\text{(Friedrichs)}\\
        \Hcal^k(M) &= \Hcal^k_N(M) \oplus (\Hcal^k(M) \cap \dif \Omega^{k-1}(M)) &&\text{(Friedrichs)}.
    \end{align*}
\end{thm}

Consider the \emph{de Rham cohomology groups relative to the boundary}
\[
    H^k_{dR}(M,\de M) \coloneq \frac{\{\omega \in \Omega^k_D(M) \colon \dif \omega = 0\}}{\{ \dif \eta \colon \eta \in \Omega^{k-1}_D(M)\}}
\]
(by Lemma~\ref{lemma: dOmega_D subset Omega_D}, the definition is well posed).

\begin{thm}\label{thm: Hodge - de Rham with Dirichlet BC}
    The quotient map induces an isomorphism 
    \[
        \Hcal^k_D(M) \xrightarrow{\sim} H^k_{dR}(M,\de M).
    \]
    In other words, every de Rham cohomology class relative to the boundary is represented by a unique harmonic form satisfying the homogeneous Dirichlet boundary condition.
\end{thm}
\begin{proof}
    Let's prove that \(\Hcal^k_D(M) \to H^k_{dR}(M,\de M)\) is injective. Suppose \(\omega = \dif \theta \in \Hcal^k_D(M)\) for some \(\theta \in \Omega^{k-1}_D(M)\). By definition, \(\omega\) is co-closed. Then, using Lemma~\ref{lem: by parts with boundary},
    \[
        (\omega,\omega)_{L^2} = (\dif \theta, \omega)_{L^2} = (\theta, \delta \omega)_{L^2} + \int_{\de M} \theta \wedge \star\omega =0,
    \]
    so \(\omega = 0\).

    Now let's prove that \(\Hcal^k_D(M) \to H^k_{dR}(M,\de M)\) is surjective. Fix a class \([\omega] \in H^k_{dR}(M,\de M)\) and, using Theorem~\ref{thm: Hodge with BC},  write 
    \[
        \omega = \theta + \dif \eta + \delta (\alpha + \beta)
    \]
    with \(\theta \in \Hcal^k_D(M), \eta \in \Omega^{k-1}_D(M), \alpha \in \Omega^{k+1}_N(M)\) and \(\beta \in \Omega^{k+1}(M)\) such that \(\delta \beta \in \Hcal^k(M)\). Since \(0=\dif \omega = \dif \delta \alpha\) and \(\alpha \in \Omega^{k+1}_N(M)\), by Lemma~\ref{lem: by parts with boundary}
    \[
        (\delta \alpha, \delta \alpha)_{L^2} = (\alpha, \dif \delta \alpha)_{L^2} - \int_{\de M}\delta \alpha \wedge \star \alpha =0,
    \]
    so actually \(\omega = \theta + \dif \eta + \delta \beta\). But then \(\delta \beta \in \Hcal^k_D(M)\), therefore \(\delta \beta = 0\). Thus, \(\omega = \theta + \dif \eta\) with \(\eta \in \Omega^{k-1}_D(M)\), i.e., \([\omega]=[\theta]\).
\end{proof}
