Denote \(\Omega^k=\Omega^k(M)\). The Hodge theorem is about the relation between the topology of \(M\) and the presence of (Hodge) harmonic forms.
\begin{defn}
    A form \(\omega\in \Omega^k\) is \emph{(Hodge) harmonic} if \(\triangle \omega = 0\). We will denote by
    \[
        \Hcal^k=\Hcal^k(M,g) \coloneq \{\omega \in \Omega^k \colon \triangle \omega = 0\}
    \]
    the subspace of harmonic \(k\)-forms. 
\end{defn}


\begin{lemma}\label{lemma: harmonic iff closed and co-closed}
    A form \(\omega \in \Omega^k\) is harmonic if and only if is both closed and co-closed. In other words, 
    \[
        \Hcal^k = \ker \dif|_{\Omega^k} \cap \ker \delta|_{\Omega^{k+1}}. 
    \]
\end{lemma}
\begin{proof}
    Just observe that for any \(\omega \in \Omega^k\),
    \[
        (\triangle\omega,\omega)_{L^2} = (\dif \omega, \dif \omega)_{L^2} + (\delta \omega, \delta\omega)_{L^2}.\qedhere
    \]  
\end{proof}

\begin{thm}[Hodge decomposition]\label{thm: Hodge}
    The space \(\Hcal^k\) is finite dimensional and it holds the \(L^2\)-orthogonal decomposition
    \[
        \Omega^k= \Hcal^k \oplus \dif \Omega^{k-1} \oplus \delta \Omega^{k+1}.
    \]
    In particular, every \(\omega \in \Omega^k\) can be written uniquely as 
    \[
        \omega = \alpha + \dif \beta + \delta \gamma,
    \]
    with \(\alpha \in \Hcal^k\) harmonic, \(\beta \in \Omega^{k-1}\) and \(\gamma \in \Omega^{k+1}\). Moreover, we can choose \(\beta\) to be co-exact and \(\gamma\) to be exact.
\end{thm}
\begin{proof}
    The classical Fredholm alternative for elliptic operators yields that \(\Hcal^k\) is finite dimensional and the orthogonal decomposition
    \[
        L^2(M;\Lambda^kT^*M)= \ker(\triangle^*) \oplus \im(\triangle)= \ker(\triangle) \oplus \im(\triangle),
    \]
    having used that \(\triangle\) is self-adjoint. Then, intersecting with \(\Omega^k\) and applying elliptic regularity, we get
    \[
        \Omega^k = \Hcal^k \oplus \triangle \Omega^k.
    \]
    
    Clearly, \(\triangle \Omega^k = \dif \Omega^{k-1} + \delta \Omega^{k+1}\), as
    \[
        \triangle \eta = \dif (\delta \eta) + \delta (\dif \eta) \qquad \forall \eta \in \Omega^k.
    \]
    Moreover, \(\dif\Omega^{k-1}\) and \(\delta\Omega^{k+1}\) are orthogonal. Indeed, for any \(\beta \in \Omega^{k-1}\) and \(\gamma \in \Omega^{k+1}\),
    \[
        g(\dif \beta, \delta \gamma) = g(\dif^2\beta, \gamma) =0. \qedhere
    \]
\end{proof}


Let us denote 
\[
    H^k_{dR}(M) \coloneq \frac{\ker \dif|_{\Omega^k}}{\dif\Omega^{k-1}}
\]
the de Rham cohomology groups.

\begin{thm}\label{thm: Hodge - de Rham}
    The canonical map \(\Hcal^k \to H^k_{dR}(M)\) is an isomorphism. In other words, every de Rham cohomology class is represented by a unique harmonic form.
\end{thm}
\begin{proof}
    Let's prove that \(\Hcal^k \to H^k_{dR}(M)\) is injective. Suppose \(\omega = \dif \theta\) is harmonic. In particular, it's co-closed, that is, \(\delta \dif \theta = 0\). Then 
    \[
        (\omega,\omega)_{L^2} = (\theta, \delta \dif \theta)_{L^2}=0,
    \]
    so \(\omega = 0\).

    Now let's prove that \(\Hcal^k \to H^k_{dR}(M)\) is surjective. Fix a class \([\omega] \in H^k_{dR}(M)\) and write 
    \[
        \omega = \alpha + \dif \beta + \delta \gamma
    \]
    with \(\alpha \in \Hcal^k\). Since \(0=\dif \omega = \dif \delta \gamma\),
    \[
        (\delta \gamma, \delta \gamma)_{L^2} = (\gamma, \dif \delta \gamma)_{L^2}=0,
    \]
    so actually \(\omega = \alpha + \dif \beta\). Thus \([\omega]=[\alpha]\).
\end{proof}

\begin{prop}[Bochner]\label{prop: Bochner}
    If \((M,g)\) has non negative Ricci curvature, then every harmonic 1-form is parallel.
\end{prop}
\begin{proof}
    For every \(\omega \in \Hcal^1\), by Lemma~\ref{lemma: Weitzenbock identity} and the computation in Remark~\ref{rmk: Weitzenbock curvature operator on 1-form},
    \[
        0 = g(\triangle \omega, \omega) = g(\nabla^*\nabla \omega, \omega) + \Ric(\omega^\sharp,\omega^\sharp).
    \]
    Integrating,
    \[
        0 = \int_M g(\nabla \omega, \nabla \omega) + \int_M\Ric(\omega^\sharp,\omega^\sharp) \ge 0,
    \]
    thus \(\nabla \omega = 0\).
\end{proof}
\begin{cor}
    If \(\Ric\ge 0\) and it is positive at one point, then \(b_1(M)=0\).
\end{cor}
\begin{proof}
    Suppose by contradiction that \(b_1(M)>0\), so by Theorem~\ref{thm: Hodge - de Rham} there is a non trivial harmonic 1-form \(\omega \in \Hcal^1\). By Proposition~\ref{prop: Bochner}, \(\omega\) is parallel, so \(\omega|_p \ne 0 \) for every \(p \in M\). By the computation in the proof of Proposition~\ref{prop: Bochner},    
    \[
        0 = \int_M g(\nabla \omega, \nabla \omega) + \int_M\Ric(\omega^\sharp,\omega^\sharp) > 0,
    \]
    a contradiction.
\end{proof}

\begin{cor}
    If \(\Ric = 0\), then \(b_1(M)=\dim H^1_{dR}(M)\le n\), with equality if and only if \((M,g)\) is a flat torus.
\end{cor}
\begin{proof}
    Since \(\Ric \ge 0\), by Proposition~\ref{prop: Bochner} every harmonic 1-form is parallel, so the evaluation \(\Hcal^1 \to T_p^*M \) is an injective linear homomorphism. In particular, by Theorem~\ref{thm: Hodge - de Rham},
    \[
        b_1(M) = \dim \Hcal^1 \le n.
    \]
    
    If equality holds, then a basis of \(\Hcal^1\) whose evaluation at a point gives an orthonormal frame gives rise to a global parallel orthonormal frame \(\{\omega_1,\dots,\omega_n\}\) of \(T^*M\). This implies that \((M,g)\) is flat. Then, the universal cover is \(\widetilde{M}= \R^n\) with flat metric. Pulling back the orthonormal frame given by \(e_i = (\omega^i)^\sharp\) to \(\widetilde{M}=\R^n\), the frame \(\{\widetilde{e}_1,\dots, \widetilde{e}_n\}\) remains parallel, hence the \(\widetilde{e}_i\)'s are the usual cartesian coordinate vector fields. Moreover, they are invariant under the action of \(\pi_1(M)\) on \(\R^n\), which means that \(\pi_1(M)\) acts by translations. Since \(\pi_1(M)\) is finitely generated, \(\pi_1(M) = \Z^m\) for some \(m\). Since the quotient \(M=\R^n/\pi_1(M)\) is compact and the action is free, it must be \(m=n\), so \(M\) is a torus.
\end{proof}