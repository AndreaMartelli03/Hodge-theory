\documentclass{article}[12pt]

%----- Math ---------------------------------------------
\usepackage{amsmath}   
\usepackage{mathrsfs}      
\usepackage{mathtools}       
\usepackage{amssymb}   
\usepackage{amsthm}   
\usepackage{esint}     
\usepackage{resmes} 
\usepackage{stackengine}
\usepackage{amsfonts}
\usepackage{stmaryrd}
\usepackage{dsfont}



%----- Design ------------------------------------------- 
\usepackage{lastpage}                              
\usepackage{enumitem}
\usepackage{multirow}


%----- Intestazioni ---------------------------------------
\usepackage{fancyhdr}
\pagestyle{fancy}
\fancyhf{} % Clear all headers and footers
\fancyhead[R]{\nouppercase{\leftmark}} % Header right on all pages
\fancyfoot[C]{\thepage} % Footer center, page number
\setlength{\headheight}{14.5pt}

% Pacchetto titlesec per personalizzare i titoli dei capitoli
\usepackage{titlesec}

% Comando per settare l'intestazione correttamente per l'introduzione
\newcommand{\chapterwithintroduction}[1]{
	\chapter*{#1}
	\addcontentsline{toc}{chapter}{#1}
	\markboth{#1}{}
}

%----- Pacchetti Disegno --------------------------------
\usepackage{tikz}
\makeatother 
\usetikzlibrary{3d,perspective}
\usetikzlibrary{patterns}
\usetikzlibrary{arrows,calc,patterns}
\usepackage{tikz-cd}
\usepackage{graphicx}
\usepackage{thmtools}
\usepackage{xcolor}
\usepackage[all]{xy}        
\usetikzlibrary{decorations.markings}
\usetikzlibrary{hobby}
\usepackage{pgfplots}
\pgfplotsset{compat=1.18}


%----- Symbols -----------------------------------------
% ------ CAPITAL MATHBB --------------------
\providecommand{\B}{\mathbb{B}}
\providecommand{\C}{\mathbb{C}}
\providecommand{\D}{\mathbb{D}}
\providecommand{\E}{\mathbb{E}}
\providecommand{\F}{\mathbb{F}}
\providecommand{\N}{\mathbb{N}}
\providecommand{\Pbb}{\mathbb{P}}
\providecommand{\Q}{\mathbb{Q}}
\providecommand{\R}{\mathbb{R}}
\providecommand{\Sp}{\mathbb{S}}
\providecommand{\T}{\mathbb{T}}
\providecommand{\Z}{\mathbb{Z}}

%------- CAPITAL MATHCAL --------
\providecommand{\Tcal}{\mathcal{T}}
\providecommand{\Lcal}{\mathcal{L}}
\providecommand{\Mcal}{\mathcal{M}}
\providecommand{\Acal}{\mathcal{A}}
\providecommand{\Hcal}{\mathcal{H}}



%-------- GREEK -----------
\providecommand{\eps}{\varepsilon}
\providecommand{\vp}{\varphi}

%-------- STANDARD MATH ---------
\providecommand{\Fam}{\mathscr{F}}
\providecommand{\id}{\mathrm{id}}
\providecommand{\Id}{\mathrm{Id}}
\providecommand{\obar}[1]{\overline{#1}}
\DeclareMathOperator{\re}{Re}
\DeclareMathOperator{\im}{Im}
\DeclareMathOperator{\val}{val}
\DeclareMathOperator{\Ind}{Ind}
\DeclareMathOperator{\ind}{ind}
\DeclareMathOperator{\sgn}{sgn}


%-------- ARROWS ---------
\providecommand{\mapsfrom}{\mathrel{\reflectbox{\ensuremath{\mapsto}}}}
\providecommand{\weakto}{\rightharpoonup}
\providecommand{\weakstarto}{\mathrel{\ensurestackMath{\stackon[1pt]{\rightharpoonup}{\scriptstyle\ast}}}}


%-------- CALCULUS ----------------
\providecommand{\de}{\partial}
\providecommand{\dif}{\mathrm{d}}
\providecommand{\acosh}{\mathrm{cosh}^{-1}}
\DeclareMathOperator{\dist}{dist}
\DeclareMathOperator{\diam}{diam}
\DeclareMathOperator{\gr}{graph}
\DeclareMathOperator{\dive}{div}
\DeclareMathOperator{\curl}{curl}
\DeclareMathOperator{\supp}{supp}
\DeclareMathOperator{\spt}{spt}
\DeclareMathOperator{\Conv}{Conv}
\DeclareMathOperator{\Hess}{Hess}
\DeclareMathOperator{\grad}{grad}
\DeclareMathOperator{\Crit}{Crit}
\DeclareMathOperator{\Reg}{Reg}

%--------- TOPOLOGY ---------
\DeclareMathOperator{\ext}{ext}
\DeclareMathOperator{\inte}{int}
\providecommand{\clos}[1]{\overline{#1}}

%--------- ALGEBRA ------
\DeclareMathOperator{\End}{End}
\DeclareMathOperator{\Hom}{Hom}
\DeclareMathOperator{\rank}{rank}
\DeclareMathOperator{\tr}{tr}
\DeclareMathOperator{\trace}{trace}
\DeclareMathOperator{\Span}{Span}
\DeclareMathOperator{\coker}{coker}

%-------- DIFFERENTIAL GEOMETRY ----
\providecommand{\sff}{\mathrm{I\!I}}
\DeclareMathOperator{\area}{area}
\DeclareMathOperator{\Area}{Area}
\DeclareMathOperator{\len}{length}
\DeclareMathOperator{\Len}{Length}
\DeclareMathOperator{\Ric}{Ric}
\DeclareMathOperator{\Riem}{Riem}
\DeclareMathOperator{\Sec}{Sec}
\DeclareMathOperator{\Scal}{Scal}
\DeclareMathOperator{\Rscal}{R}

%--------- MEASURE THEORY -----------
\providecommand{\Haus}{\mathscr{H}}
\providecommand{\Leb}{\mathscr{L}}
\usepackage{resmes}
\DeclareMathOperator{\essup}{essup}

%----- Hyperref ------------------------------------------
\usepackage{nameref}
\usepackage{csquotes}
\usepackage{hyperref}
\hypersetup{
	colorlinks=true,
	linkcolor=black, % Color for normal internal links
	filecolor=black, % Color for file links
	urlcolor=blue, % Color for external links
	citecolor=blue % Color for citations
}
\usepackage{cleveref} 


% -----Ambienti matematici-------------------------------
\theoremstyle{plain}
\newtheorem{thm}{Theorem}[section]
\newtheorem*{thm*}{Theorem}
\newtheorem{prop}[thm]{Proposition}
\newtheorem{cor}[thm]{Corollary}
\newtheorem{lemma}[thm]{Lemma}
\newtheorem{claim}{Claim}

\theoremstyle{definition}
\newtheorem{defn}[thm]{Definition}
\newtheorem{axiom}{Axiom}

\theoremstyle{remark}
\newtheorem{rmk}[thm]{Remark}
\newtheorem{ex}[thm]{Example}
\newtheorem{exercise}{Exercise}


%------BIBLIOGRAFIA-----------
\usepackage[style=alphabetic, backend=biber]{biblatex}
% Definizione di un nuovo driver per nascondere URL e DOI
\AtEveryBibitem{
 	\clearfield{url} % Nasconde l'URL
 	\clearfield{doi} % Nasconde il DOI
 	\clearfield{isbn} % Nasconde ISBN
 	\clearfield{issn} % Nasconde ISSN
 	\clearfield{note} % Nasconde le note
}
\addbibresource{biblio.bib}

\title{Hodge theory and stationary Maxwell equations}
\author{Andrea Martelli}
\date{September 2025}
\begin{document}

\maketitle

\begin{abstract}
    These notes are the content of my seminar for the exam in the course \emph{Mathematical Physics – Differential Geometric Methods}, taught by Enrico Pagani during the Spring semester of 2025.
\end{abstract}
\tableofcontents

\section*{Introduction}
The main purposes of these notes are
\begin{itemize}
    \item to give a quick introduction to Hodge theory on oriented closed Riemannian manifolds;
    \item to analyze the relations between the topology of a domain and a boundary value problem on it (the stationary Maxwell equations in vacuum with homogeneous Dirichlet boundary condition).
\end{itemize}
The two goals are very different, because they have different geometric settings. Nevertheless, it's possible to build a theory analogous to the Hodge theory for manifolds with boundary, when imposing homogeneous Dirichlet boundary conditions. 

Therefore, in the first section I introduce the Hodge star operator without assuming anything on the manifold, other than it is oriented and compact (however, one could do everything with compactly supported forms and remove also the latter assumption). In the second section, I assume that the manifold is without boundary and treat Hodge theory as done in \cite{PetersenRiemannian_2006}. In particular, I prove some result concerning the first Betti number of closed orientable manifolds admitting non negative Ricci curvature. In the third section, I present the analogous of Hodge theory for manifolds with boundary, without proving the main result (Theorem~\ref{thm: Hodge with BC}). A detailed reference for this topic is \cite{SchwarzHodge_1995}. In the fourth and last section, I analyze a boundary value problem associated to the stationary Maxwell equation in vacuum depending on the topology of the domain in \(\R^3\). I find this discussion quite interesting, because it shows that topology can obstruct both existence and uniqueness of the solution to the boundary value problem.

\section{Hodge star operator}
Let \((M,g)\) be an oriented Riemannian \(n\)-manifold and denote by \(\omega_g \in \Omega^n(M)\) the corresponding Riemannian volume form. 

Recall that the metric tensor \(g\) induces a scalar product (still denoted by \(g\)) on each fiber \((T^*_pM)^{\otimes k}\) by letting
\[
    g(\alpha,\beta) = g^{i_1j_1}\cdots g^{i_kj_k}\alpha_{i_1 \dots i_k} \beta_{j_1 \dots j_k}
\]
if \(\alpha = \alpha_{i_1 \dots i_k}\dif x^{i_1}\otimes \cdots \otimes\dif x^{i_k}\) and \(\beta = \beta_{j_1 \dots j_k}\dif x^{j_1}\otimes \dots \otimes \dif x^{j_k} \) in local coordinates. This scalar product restricts also to the subspace \(\Lambda^kT^*_pM\), and actually, since for an orthonormal dual basis \(\{e^1,\dots,e^n\}\)
\[
\begin{split}
    g(e^{i_1} \wedge \cdots \wedge e^{i_k},e^{i_1} \wedge&\cdots \wedge e^{i_k}) \\
    &= \frac{1}{(k!)^2}\sum_{\sigma,\tau \in S_k}\sgn(\sigma)\sgn(\tau)\delta^{\sigma(i_1)\tau(i_1)} \cdots \delta^{\sigma(i_k)\tau(i_k)} \\
    &= \frac{1}{(k!)^2}\sum_{\sigma \in S_k}\sgn(\sigma)^2 = \frac{1}{k!},
\end{split}
\]
(and \(g(e^I, e^J)=0\) if \(I\ne J\), of course) we have for any \(\omega = \omega_{i_1 \dots i_k}e^{i_1} \wedge \cdots \wedge e^{i_k}\) and \(\eta_{j_1 \dots j_k}e^{j_1} \wedge \cdots \wedge e^{j_k} \in \Lambda^kT^*_pM\)
\[
    g(\omega,\eta) = \frac{1}{k!}\delta^{i_1j_1}\cdots \delta^{i_kj_k}\omega_{i_1 \dots i_k} \eta_{j_1 \dots j_k} = \frac{1}{k!}g(\overline{\omega},\overline{\eta})
\]
where
\[
\begin{split}
    \overline{\omega} &= \omega_{i_1 \dots i_k} e^{i_1} \otimes \cdots \otimes e^{i_k}\\
    \overline{\eta} &= \eta{j_1 \dots j_k} e^{j_1} \otimes \cdots \otimes e^{j_k}
\end{split}
\]
(and the indices are not ordered in this sum).


We can identify \((\Lambda^kT^*_pM)^*\) with \(\Lambda^kT^*_pM\) by lowering every index, that is via the linear isomorphism
\begin{align*}
    \flat \colon \Lambda^kT^*_pM &\to (\Lambda^{k}T^*_pM)^* \\
    \omega &\mapsto \omega^\sharp = g(\omega, \cdot ).
\end{align*}
We denote by \(\flat\) the inverse of \(\sharp\).

On the other hand, the wedge product induces a pairing
\begin{align*}
    \wedge \colon \Lambda^kT^*_pM \times \Lambda^{n-k}T^*_pM &\to \R\\
    (\omega,\eta) &\mapsto g(\omega\wedge \eta, \omega_g|_p)
\end{align*}
that allows us to identify \(\Lambda^{n-k}T^*_pM \equiv (\Lambda^kT^*_pM)^*\). 

Combining these two identifications, we can give the following definition. 
\begin{defn}
    The \emph{Hodge star operator} is the \(C^\infty(M)\)-linear isomorphism 
    \[
        \star \colon \Omega^k(M) \to \Omega^{n-k}(M)
    \]
    defined by
    \[
        g(\eta, \star\omega) = g(\eta \wedge \omega, \omega_g) \qquad \forall \eta \in \Omega^k(M).
    \]
\end{defn}

\begin{lemma}
    Let \(\omega, \eta \in \Omega^k(M)\).
    \begin{enumerate}[label=(\alph*)]
        \item \(\omega \wedge \star \eta = \eta \wedge \star \omega\).
        \item \(\star \star \omega = (-1)^{k(n-k)} \omega\), i.e., \(\star \star = (-1)^{k(n-k)}\Id\).
        \item \(g(\star \omega, \star \eta) = g(\omega,\eta)\), i.e., \(\star\) is an isometry. 
    \end{enumerate}
\end{lemma}
\begin{proof}
    Point \((a)\) follows from the symmetry of scalar product:
    \[
        \omega \wedge \star \eta = g(\omega,\eta)\omega_g = \eta \wedge \star \omega.
    \]
    To prove point \((b)\), by linearity, it's enough to prove it for elements of the form \(\dif x^{i_1} \wedge \cdots \wedge \dif x^{i_k}\). So let \(\{i_1,\dots,i_k,j_1,\dots,j_{n-k}\} = \{1,\dots,n\}\) with \(i_1 < \dots < i_k\) and \(j_1 < \dots < j_{n-k}\). Then
    \[
        \star (\dif x^{i_1} \wedge \cdots \wedge \dif x^{i_k}) = \frac{\sigma(i_1 \dots i_k j_1 \dots j_{n-k})}{\sqrt{\det(g_{ij})}}\dif x^{j_1} \wedge \cdots \wedge \dif x^{j_k}
    \]
    where
    \[
    \sigma(i_1 \dots i_k j_1 \dots j_{n-k}) = (-1)^{i_1 + \cdots + i_k - (1+\cdots + k)}
    \]
    is the sign of the corresponding permutation.
    Thus,
    \[
        \star \star (\dif x^{i_1} \wedge \cdots \wedge \dif x^{i_k}) =  \frac{(-1)^{1+\cdots+n-(1+\cdots+k) - (1 +\cdots+(n-k))}}{\sqrt{\det(g_{ij})}}\dif x^{i_1} \wedge \cdots \wedge \dif x^{i_k}
    \]
    Then observe that
    \[        
    1+\cdots+n-(1+\cdots+k) - (1 +\cdots+(n-k)) = k(n-k).
    \]
    
    Finally, point \((c)\) follows from \((b)\):
    \[
    g(\star \omega, \star \eta) \omega_g = \star \omega \wedge \star \star \eta = (-1)^{k(n-k)} \star \omega \wedge \eta = \eta \wedge \star \omega = g(\eta,\omega)\omega_g. \qedhere
    \]
\end{proof}






\section{Hodge theory for oriented closed manifolds}
\subsection{Co-differential and Laplacians}
Assume that \((M,g)\) is also closed (compact and without boundary). From now on, we will omit the Riemannian volume form when integrating functions, that is, we write
\[
    \int_M f \coloneq \int_Mf \omega_g.
\]
We can endow \(\Omega^k(M)\) with the scalar product
\[
    (\omega,\eta)_{L^2} \coloneq \int_M g(\omega,\eta) = \int_M \omega \wedge \star \eta.
\]
Taking the metric completion with respect to the induced norm, we construct the Hilbert space of \(L^2\) sections of \(\Lambda^kT^*M\), which we will denote by \(L^2(M;\Lambda^kT^*M)\). Observe that since the Hodge star is a linear isometry pointwise, it's also a linear isometry between the corresponding \(L^2\)-spaces.

Similarly, defining the scalar product
\[
    (\omega,\eta)_{H^m} \coloneq \int_M g(\omega,\eta) + \sum_{j=1}^m \int_M g(\nabla^j\omega,\nabla^j\eta) 
\]
and taking the metric completion of \(\Omega^k(M)\) with respect to the induced norm, we find the Sobolev spaces \(H^m(M;\Lambda^kT^*M)\).

Observe that for any \(\eta \in \Omega^k(M)\),
\[
    \|\dif \eta\|_{L^2} \le \|\nabla \eta\|_{L^2} \le \|\eta\|_{H^1}.
\]
Indeed, at the center of normal coordinates,
\[
    \dif \eta = \de_j\eta_{i_1\dots i_k}\dif x^j \wedge \dif x^{i_1}\wedge \cdots \wedge \dif x^{i_k} = \dif x^j \wedge \nabla_j\eta 
\]
so,
\[
    g(\dif \eta,\dif \eta) \le g(\nabla \eta, \nabla \eta).
\]
Thus, the exterior derivative extends to a bounded linear operator
\[
    \dif \colon H^1(M;\Lambda^kT^*M) \to L^2(M,\Lambda^{k-1}T^*M).
\]
Moreover, for fixed \(\omega \in L^2(M;\Lambda^kT^*M)\), the linear operator
\[
    \eta \mapsto \int_Mg(\omega, \dif \eta)
\]
is \(H^1\)-bounded, and can therefore be represented uniquely by an element in \(H^1(M;\Lambda^*T^{k-1}M)\).
\begin{defn}
    The \emph{co-differential} \(\delta = \dif^*\) is the adjoint of the exterior derivative, i.e., the linear mapping
    \[
        \delta \colon L^2(M;\Lambda^kT^*M) \to H^1(M;\Lambda^{k-1}T^*M)
    \]
    defined by
    \[
        \int_Mg(\delta \omega,\eta) = \int_Mg(\omega, \dif \eta) \qquad \forall \eta \in \Omega^{k-1}(M)
    \]
    for every \(\omega \in L^2(M;\Lambda^kT^*M)\).
\end{defn}
\begin{prop}\label{prop: co-boundary}
    For every \(\omega \in L^2(M;\Lambda^kT^*M)\), 
    \[
        \delta \omega = (-1)^{kn+n+1}\star \dif \star \omega.
    \]
    In particular, if \(\omega \in \Omega^k(M)\) then \(\delta \omega \in \Omega^{k-1}(M)\).
\end{prop}
\begin{proof}
    Let us prove the claim for \(\omega \in \Omega^k(M)\) (then the general claim follows by density). By Stokes theorem and the properties of \(\star\), for every \(\eta \in \Omega^{k-1}(M)\),
    \begin{align*}
        (\omega, \dif \eta)_{L^2} &= \int_M \dif \eta \wedge \star \omega \\
        &= \int_M \dif (\eta \wedge \star \omega) -(-1)^{k-1}\int_M \eta \wedge \dif \star \omega \\
        &= (-1)^k \int_M \eta \wedge \dif \star \omega \\
        &= (-1)^{k+(n-k+1)(k-1)} \int_M \eta \wedge \star \star \dif \star \omega \\
        &= (-1)^{kn+n+1}\int_M \eta \wedge \star \star \dif \star \omega = ((-1)^{kn+n+1}\star \dif \star \omega,\eta)_{L^2}. \qedhere
    \end{align*}
\end{proof}

\begin{defn}
    The \emph{Hodge Laplacian} is the second order differential operator 
    \[  
        \triangle = \delta \dif + \dif \delta.
    \]
\end{defn}

\begin{defn}
    The \emph{connection Laplacian} is the second order differential operator \(\nabla^*\nabla\) defined by
    \[
        \int_M g(\nabla^*\nabla \omega, \eta) = \int_M (\nabla\omega,\nabla \eta) \qquad \forall \eta \in \Omega^k(M)
    \]
    for any \(\omega \in \Omega^k(M)\).
\end{defn}

Both operators are self-adjoint with respect to the \(L^2\)-product: for every \(\omega,\eta \in \Omega^k(M)\), we have
\[
\begin{split}
    (\triangle \omega,\eta)_{L^2}&=\int_M g(\triangle \omega, \eta) = \int_M g(\dif \omega, \dif \eta) + \int_M g(\delta \omega,\delta \eta)\\
    (\nabla^*\nabla \omega,\eta)_{L^2}&=\int_M g(\nabla^*\nabla \omega, \eta) = \int_M (\nabla\omega,\nabla \eta)
\end{split}
\]
and both identities are symmetric on the right hand side. Moreover, for functions (\(k=0\)), since \(\nabla = \dif\) and \(\delta = 0\), they coincide to the usual Laplace-Beltrami operator (up to sign). For order higher then zero they don't coincide. However, there is a relation which involves curvature terms (the proof is just a computation, see \cite[Theorem~9.4.1]{PetersenRiemannian_2006}).

\begin{lemma}[Weitzenb\"ock identity]\label{lemma: Weitzenbock identity}
    For any \(\omega \in \Omega^k(M)\),
    \[
        \triangle\omega = \nabla^*\nabla \omega +  \Ric(\omega),
    \]
    where, for any orthonormal frame \(\{e_1,\dots,e_n\}\)
    \[
        \Ric(\omega)(X_1,\dots,X_k) \coloneq\sum_{i=1}^k\sum_{j=1}^n (R(e_j,X_i)\omega)(X_1,\dots,e_j,\dots,X_k)
    \]
    is the \emph{Weitzenb\"ock curvature operator}\footnote{
    Following the convention used in \cite{PetersenRiemannian_2006}, here
    \[
        R(X,Y)=\nabla_X\nabla_y - \nabla_Y\nabla_X-\nabla_{[X,Y]}.
    \]
    }.
\end{lemma}
\begin{rmk}\label{rmk: Weitzenbock curvature operator on 1-form}
    If \(\omega \in \Omega^1(M)\) is a 1-form, then
    \[
    \begin{split}
        g(\Ric(\omega),\omega) &= \Ric(\omega)(\omega^\sharp) = \sum_j (R(e_j,\omega^\sharp)\omega)(e_j) = \sum_j g(R(e_j,\omega^\sharp)\omega^\sharp,e_j) \\
        &= \Ric(\omega^\sharp,\omega^\sharp).
    \end{split}
    \]
\end{rmk}
\subsection{The Hodge theorem}
Denote \(\Omega^k=\Omega^k(M)\). The Hodge theorem is about the relation between the topology of \(M\) and the presence of (Hodge) harmonic forms.
\begin{defn}
    A form \(\omega\in \Omega^k\) is \emph{(Hodge) harmonic} if \(\triangle \omega = 0\). We will denote by
    \[
        \Hcal^k=\Hcal^k(M,g) \coloneq \{\omega \in \Omega^k \colon \triangle \omega = 0\}
    \]
    the subspace of harmonic \(k\)-forms. 
\end{defn}


\begin{lemma}\label{lemma: harmonic iff closed and co-closed}
    A form \(\omega \in \Omega^k\) is harmonic if and only if is both closed and co-closed. In other words, 
    \[
        \Hcal^k = \ker \dif|_{\Omega^k} \cap \ker \delta|_{\Omega^{k+1}}. 
    \]
\end{lemma}
\begin{proof}
    Just observe that for any \(\omega \in \Omega^k\),
    \[
        (\triangle\omega,\omega)_{L^2} = (\dif \omega, \dif \omega)_{L^2} + (\delta \omega, \delta\omega)_{L^2}.\qedhere
    \]  
\end{proof}

\begin{thm}[Hodge decomposition]\label{thm: Hodge}
    The space \(\Hcal^k\) is finite dimensional and it holds the \(L^2\)-orthogonal decomposition
    \[
        \Omega^k= \Hcal^k \oplus \dif \Omega^{k-1} \oplus \delta \Omega^{k+1}.
    \]
    In particular, every \(\omega \in \Omega^k\) can be written uniquely as 
    \[
        \omega = \alpha + \dif \beta + \delta \gamma,
    \]
    with \(\alpha \in \Hcal^k\) harmonic, \(\beta \in \Omega^{k-1}\) and \(\gamma \in \Omega^{k+1}\). Moreover, we can choose \(\beta\) to be co-exact and \(\gamma\) to be exact.
\end{thm}
\begin{proof}
    The classical Fredholm alternative for elliptic operators yields that \(\Hcal^k\) is finite dimensional and the orthogonal decomposition
    \[
        L^2(M;\Lambda^kT^*M)= \ker(\triangle^*) \oplus \im(\triangle)= \ker(\triangle) \oplus \im(\triangle),
    \]
    having used that \(\triangle\) is self-adjoint. Then, intersecting with \(\Omega^k\) and applying elliptic regularity, we get
    \[
        \Omega^k = \Hcal^k \oplus \triangle \Omega^k.
    \]
    
    Clearly, \(\triangle \Omega^k = \dif \Omega^{k-1} + \delta \Omega^{k+1}\), as
    \[
        \triangle \eta = \dif (\delta \eta) + \delta (\dif \eta) \qquad \forall \eta \in \Omega^k.
    \]
    Moreover, \(\dif\Omega^{k-1}\) and \(\delta\Omega^{k+1}\) are orthogonal. Indeed, for any \(\beta \in \Omega^{k-1}\) and \(\gamma \in \Omega^{k+1}\),
    \[
        g(\dif \beta, \delta \gamma) = g(\dif^2\beta, \gamma) =0. \qedhere
    \]
\end{proof}


Let us denote 
\[
    H^k_{dR}(M) \coloneq \frac{\ker \dif|_{\Omega^k}}{\dif\Omega^{k-1}}
\]
the de Rham cohomology groups.

\begin{thm}\label{thm: Hodge - de Rham}
    The canonical map \(\Hcal^k \to H^k_{dR}(M)\) is an isomorphism. In other words, every de Rham cohomology class is represented by a unique harmonic form.
\end{thm}
\begin{proof}
    Let's prove that \(\Hcal^k \to H^k_{dR}(M)\) is injective. Suppose \(\omega = \dif \theta\) is harmonic. In particular, it's co-closed, that is, \(\delta \dif \theta = 0\). Then 
    \[
        (\omega,\omega)_{L^2} = (\theta, \delta \dif \theta)_{L^2}=0,
    \]
    so \(\omega = 0\).

    Now let's prove that \(\Hcal^k \to H^k_{dR}(M)\) is surjective. Fix a class \([\omega] \in H^k_{dR}(M)\) and write 
    \[
        \omega = \alpha + \dif \beta + \delta \gamma
    \]
    with \(\alpha \in \Hcal^k\). Since \(0=\dif \omega = \dif \delta \gamma\),
    \[
        (\delta \gamma, \delta \gamma)_{L^2} = (\gamma, \dif \delta \gamma)_{L^2}=0,
    \]
    so actually \(\omega = \alpha + \dif \beta\). Thus \([\omega]=[\alpha]\).
\end{proof}

\begin{prop}[Bochner]\label{prop: Bochner}
    If \((M,g)\) has non negative Ricci curvature, then every harmonic 1-form is parallel.
\end{prop}
\begin{proof}
    For every \(\omega \in \Hcal^1\), by Lemma~\ref{lemma: Weitzenbock identity} and the computation in Remark~\ref{rmk: Weitzenbock curvature operator on 1-form},
    \[
        0 = g(\triangle \omega, \omega) = g(\nabla^*\nabla \omega, \omega) + \Ric(\omega^\sharp,\omega^\sharp).
    \]
    Integrating,
    \[
        0 = \int_M g(\nabla \omega, \nabla \omega) + \int_M\Ric(\omega^\sharp,\omega^\sharp) \ge 0,
    \]
    thus \(\nabla \omega = 0\).
\end{proof}
\begin{cor}
    If \(\Ric\ge 0\) and it is positive at one point, then \(b_1(M)=0\).
\end{cor}
\begin{proof}
    Suppose by contradiction that \(b_1(M)>0\), so by Theorem~\ref{thm: Hodge - de Rham} there is a non trivial harmonic 1-form \(\omega \in \Hcal^1\). By Proposition~\ref{prop: Bochner}, \(\omega\) is parallel, so \(\omega|_p \ne 0 \) for every \(p \in M\). By the computation in the proof of Proposition~\ref{prop: Bochner},    
    \[
        0 = \int_M g(\nabla \omega, \nabla \omega) + \int_M\Ric(\omega^\sharp,\omega^\sharp) > 0,
    \]
    a contradiction.
\end{proof}

\begin{cor}
    If \(\Ric = 0\), then \(b_1(M)=\dim H^1_{dR}(M)\le n\), with equality if and only if \((M,g)\) is a flat torus.
\end{cor}
\begin{proof}
    Since \(\Ric \ge 0\), by Proposition~\ref{prop: Bochner} every harmonic 1-form is parallel, so the evaluation \(\Hcal^1 \to T_p^*M \) is an injective linear homomorphism. In particular, by Theorem~\ref{thm: Hodge - de Rham},
    \[
        b_1(M) = \dim \Hcal^1 \le n.
    \]
    
    If equality holds, then a basis of \(\Hcal^1\) whose evaluation at a point gives an orthonormal frame gives rise to a global parallel orthonormal frame \(\{\omega_1,\dots,\omega_n\}\) of \(T^*M\). This implies that \((M,g)\) is flat. Then, the universal cover is \(\widetilde{M}= \R^n\) with flat metric. Pulling back the orthonormal frame given by \(e_i = (\omega^i)^\sharp\) to \(\widetilde{M}=\R^n\), the frame \(\{\widetilde{e}_1,\dots, \widetilde{e}_n\}\) remains parallel, hence the \(\widetilde{e}_i\)'s are the usual cartesian coordinate vector fields. Moreover, they are invariant under the action of \(\pi_1(M)\) on \(\R^n\), which means that \(\pi_1(M)\) acts by translations. Since \(\pi_1(M)\) is finitely generated, \(\pi_1(M) = \Z^m\) for some \(m\). Since the quotient \(M=\R^n/\pi_1(M)\) is compact and the action is free, it must be \(m=n\), so \(M\) is a torus.
\end{proof}

\section{Hodge theory with homogeneous Dirichlet boundary condition}
Let \((M,g)\) be a compact oriented Riemannian manifold with non empty boundary \(\de M\). A possible definition of co-boundary as the \(L^2\)-adjoint of \(\dif\) would not give rise to a self-adjoint Laplacian, because boundary terms would appear. For this reason, we give the definition of co-boundary using Proposition~\ref{prop: co-boundary}:
\[
    \delta \coloneq (-1)^{kn+n+1}\star \dif \star.
\]

\begin{lemma}\label{lem: by parts with boundary}
    For any \(\omega \in \Omega^{k}(M)\) and \(\eta \in \Omega^{k-1}(M)\),
    \[
        \int_M g(\omega, \dif \eta) = \int_M g(\delta \omega, \eta) + \int_{\de M}(\omega \wedge \star \eta)|_{\de M}.
    \]
\end{lemma}
\begin{proof}
    Compute as in the proof of Proposition~\ref{prop: co-boundary}, but without canceling the boundary term given by Stokes theorem.
\end{proof}

\begin{defn}
    We say that \(\omega \in \Omega^k(M)\) satisfies the \emph{(homogeneous) Dirichlet boundary condition} if 
        \[
            \omega|_{\de M} =0;
        \] 
    We denote
    \[
        \Omega^k_D(M) \coloneq \{ \omega \in \Omega^k(M) \colon \omega|_{\de M}=0\}.\\
    \]
\end{defn}

\begin{rmk}
    One could also consider the \emph{co-Dirichlet boundary condition}
    \[
        (\star \omega)|_{\de M} = 0
    \]
    and analogous results hold. However, the correspondent harmonic cohomology wouldn't turn out to be isomorphic to a de Rham cohomology (cf. Theorem~\ref{thm: Hodge - de Rham with Dirichlet BC}).
\end{rmk}

From now on, drop every dependence on \(M\) on the spaces \(\Omega^k,\Omega^k_D\).

We can define the Hodge Laplacian as before by 
\[
    \triangle \coloneq \dif \delta + \delta \dif.
\] 
Denote also 
\[
    \Hcal^k_D \coloneq \{ \omega \in \Omega^k_D \colon \triangle \omega = 0\}.
\]
By Lemma~\ref{lem: by parts with boundary}, \(\triangle\) is self adjoint on \(\Omega^k_D\), so it satisfies the same property we used to prove Lemma~\ref{lemma: harmonic iff closed and co-closed}. Thus, with the same proof, we get
\[
    \Hcal^k_D = \ker \dif|_{\Omega^k_D} \cap \ker \delta|_{\Omega^k_D}.
\]

Using the same argument with the Fredholm alternative, we get the following.

\begin{thm}\label{thm: Hodge with BC}
    The space \(\Hcal^k_D\) is finite dimensional. Moreover, there hold the following \(L^2\)-orthogonal decomposition:
    \begin{equation*}
        \Omega^k_D = \Hcal^k_D \oplus \dif \Omega^{k-1}_D \oplus \delta\Omega^{k+1}_D \cap \Omega^{k}_D.
    \end{equation*}
    In particular, every \(\omega \in \Omega^k_D\) can be written uniquely as 
    \[
        \omega = \alpha + \dif \beta + \delta\gamma,
    \]
    with \(\alpha \in \Hcal^k_D\) harmonic, \(\beta \in \Omega^{k-1}_D\) and \(\gamma \in \Omega^{k+1}_D\). 
\end{thm}
\begin{proof}
    Exactly as in the proof of Theorem~\ref{thm: Hodge}, the Fredholm alternative yields that \(\Hcal^k_D\) is finite dimensional and, intersecting the decomposition with \(\Omega^k_D\) and using elliptic regularity, 
    \begin{equation*}
        \Omega^k_D = \Hcal^k_D \oplus (\triangle \Omega^k_D \cap \Omega^k_D).
    \end{equation*}

    Observe that \(\dif \Omega^{k-1}_D \subset \Omega^k_D\), because \((\dif \omega)|_{\de M} = \dif (\omega|_{\de M}) = 0\) for any \(\omega \in \Omega^{k-1}_D\). Moreover, since every harmonic form is co-closed, \(\dif \Omega^{k-1}_D \subset (\Hcal^k_D)^\perp\), so
    \[
        \triangle \Omega^k_D \cap \Omega^k_D = \dif \Omega^{k-1}_D \oplus (\dif \Omega^{k-1}_D)^\perp \cap \Omega^k_D.
    \]
    
    To conclude, it's enough to prove that 
    \begin{equation}\label{eq: boundary Hodge - 1}
        \dif \Omega^{k-1}_D = (\ker \delta|_{\Omega^k_D})^\perp
    \end{equation}
    and
    \begin{equation}\label{eq: boundary HOdge - 2}
        \ker \delta|_{\Omega^k_D} \cap \Delta \Omega^k_D = \delta \Omega^k_D \cap \Omega^k_D
    \end{equation}

    Let us prove \eqref{eq: boundary Hodge - 1}. On one hand, \(\dif \Omega^{k-1}_D \subset (\ker \delta|_{\Omega^k_D})^\perp\) because for any \(\omega \in \Omega^{k-1}_D\) adn \(\eta \in \ker \delta|_{\Omega^k_D}\),
    \[
        (\dif \omega, \eta)_{L^2} = (\omega, \delta \eta)_{L^2} = 0.
    \]
    On the other hand, if \(\omega \in (\ker \delta|_{\Omega^k_D})^\perp \subset (\Hcal^k_D)^\perp \cap \Omega^k_D = \triangle \Omega^k_D \cap \Omega^k_D\), then we can write
    \[
        \omega = \triangle \beta = \dif \delta \beta + \delta \dif \beta
    \]
    for some \(\beta \in \Omega^k_D\). But \(\dif \delta \beta \in \dif \Omega^{k-1}_D \subset (\ker \delta|_{\Omega^k_D})^\perp\), so also \(\delta \dif \beta \in (\ker \delta|_{\Omega^k_D})^\perp\), which implies that \(\delta \dif \beta =0\). Thus \(\omega = \dif \delta \beta\). This proves \((\ker \delta|_{\Omega^k_D})^\perp \subset \dif \Omega^{k-1}_D\).

    Now let's prove \eqref{eq: boundary HOdge - 2}. The inclusion \(\delta \Omega^k_D \cap \Omega^k_D \subset \ker \delta|_{\Omega^k_D} \cap \Delta \Omega^k_D\) follows immediately by the fact that \(\delta^2=0\), so let's prove the opposite inclusion. Let \(\omega\in \ker \delta|_{\Omega^k_D} \cap \Delta \Omega^k_D\), so write
    \[
        \omega = \triangle \beta = \dif \delta + \delta \dif \beta.
    \]
    But observe that since \(0=\delta \omega= \delta \dif \delta \beta\), 
    \[
        (\dif \delta \beta, \dif \delta \beta)_{L^2} = (\delta \dif \delta \eta, \eta)_{L^2} = 0,
    \]
    so actually \(\omega = \delta \dif \beta \in \delta \Omega^k_D\) (and by assumption \(\omega \in \Omega^k_D\)).
\end{proof}


Consider the \emph{de Rham cohomology groups relative to the boundary}
\[
    H^k_{dR}(M,\de M) \coloneq \frac{\ker \dif|_{\Omega^k_D}}{\dif \Omega^{k-1}_D}
\]
(observe that \(\dif \Omega^{k-1}_D \subset \Omega^k_D\) because \((\dif \omega)|_{\de M} = \dif (\omega|_{\de M}) = 0\) for any \(\omega \in \Omega^{k-1}_D\), so the definition is well posed).

With the same proof of Theorem~\ref{thm: Hodge - de Rham}, we get the following analogous result.
\begin{thm}\label{thm: Hodge - de Rham with Dirichlet BC}
    The natural map \(\Hcal^k_D \to H^k_{dR}(M,\de M)\) is a linear isomorphism. In other words, every de Rham cohomology class relative to the boundary is represented by a unique harmonic form satisfying the homogeneous Dirichlet boundary condition.
\end{thm}

\section{Stationary Maxwell equations}
Consider a region \(M \subset \R^3\) bounded by a smooth \emph{perfect conductor} \(\de M\), having outer unit normal \(\nu \colon \de M \to \R^3\). The static Maxwell equations in vacuum are
\begin{align*}
    \curl \mathbf{E} &= 0 && \text{(Faraday)}\\
    \dive \mathbf{E} &= \frac{\rho}{\eps_0} & &\text{(Gauss)}\\
    \dive \mathbf{B} &= 0 && \text{(no magnetic charges)}\\
    \curl \mathbf{B} &= \mu_0\mathbf{j} && \text{(Ampère)}
\end{align*}
with boundary conditions
\begin{equation}\label{eq: BC vectors}
    \mathbf{E} \times \nu = 0, \qquad \mathbf{B} \cdot \nu = 0  \qquad \text{on }\de M.
\end{equation}
where the electric field \(\mathbf{E} \colon M \to \R^3\) and the magnetic field \(\mathbf{B} \colon M \to \R^3\) are the unknowns, while the charge density \(\rho \in C^\infty_c(M)\), the current density \(\mathbf{j} \in C^\infty_c(M;\R^3)\), and the electric and magnetic permittivity constants in vacuum \(\eps_0,\mu_0 >0\) (that from now on we assume to be equal to 1, after having chosen a suitable unit system) are the data.

If we consider the differential forms
\begin{align*}
        E &\coloneq E_1 \dif x^1 + E_2 \dif x^2 + E_3 \dif x^3,\\
        B &\coloneq B_1 \dif x^2 \wedge \dif x^3 - B_2 \dif x^2 \wedge \dif x^3 + B_3 \dif x^1 \wedge \dif x^2\\
        J &\coloneq j_1 \dif x^1 + j_2 \dif x^2 + j_3 \dif x^3 
\end{align*}
a simple computation shows that the Maxwell equations become
\begin{align}
    \dif E = 0 && \delta E = \rho \\
    \dif B = 0 && \delta B = J
\end{align}
with boundary conditions
\begin{equation}\label{eq: BC forms}
    E|_{\de M} = 0, \qquad B|_{\de M} = 0.
\end{equation}
In fact, if we consider the simple case where \(\de M \approx \{x^3=0\}\) and \(M \approx \{x^3<0\}\), so that \(\nu \approx e_3\), then the boundary conditions \eqref{eq: BC vectors} become
\[
    E_1=E_2= 0, \qquad B_3=0.
\]
On the other hand,
\[
    E|_{\de M} = E_1 \dif x^2 + E_2 \dif x^2, \qquad B|_{\de M} = B_3 \dif x^1 \wedge \dif x^2,
\]
so the two sets of boundary conditions coincide.

Now we apply the results of the previous section to analyze topologically the system of Maxwell equations.
By Theorem~\ref{thm: Hodge with BC}, since \(E\) and \(B\) are closed
\[
    E = \alpha + \dif \phi, \qquad B = \beta + \dif A
\]
where \(\alpha \in \Hcal^1_D, \beta \in \Hcal^2_D\) and \(\phi \in \Omega^0_D, A \in \Omega^1_D\). Actually, we can assume that \(\phi\) and \(A\) are co-closed. In fact, \(\phi\) is a 0-form, so it's trivially co-closed. To fix \(A\), do a gauge transform \(A'=A+\dif \psi\), for some \(\psi \in \Omega^0_D\), so that \(\dif A'=\dif A\) and \(A' \in \Omega^1_D\). Then,
\[
    \delta A' = \delta A + \triangle \psi,
\]
so we need to solve the equation 
\[
    \triangle \psi = - \delta A
\]
with Dirichlet boundary conditions. Since \(\delta A \in (\Hcal^0_D)^\perp\), this problem have a unique solution, and we can replace \(A\) with the co-closed form \(A'\).

Now, using the inhomogeneous Maxwell equations and the co-closedness of \(\phi\) and \(A\),
\[
    \triangle\phi =  \rho, \qquad \triangle A =  J.
\]
Thus, solution to the Maxwell equations exist if and only if \(\rho \in (\Hcal^0_D)^\perp\) and \(J \in (\Hcal^1_D)^\perp\), i.e.,
\[
    \int_M \rho \vp = 0 \qquad \qquad \forall \vp \in \Hcal^0_D
\]
and
\[
    \int_M \alpha \wedge \star J = 0 \qquad \qquad \forall \alpha \in \Hcal^1_D.
\]

In our case, since \(M \subset \R^3\) is a bounded domain, by the usual elliptic theory the only solution of the boundary value problem
\begin{align*}
    \Delta u &= 0 \qquad \text{ on }M\\
    u &= 0  \qquad \text{ on } \de M
\end{align*}
is \(u=0\), so
\[
    \Hcal^0_D(M) = 0.
\]
This means that there are no conditions on the charge density.

Moreover, by Theorem~\ref{thm: Hodge - de Rham with Dirichlet BC}, \(\alpha\) and \(\beta\) determine the cohomology class relative to the boundary of the electric and the magnetic field, respectively.

Therefore, since it is the reason of the existence of non trivial harmonic forms, the presence of topology forces
\begin{itemize}
    \item compatibility conditions on the data
    \item non uniqueness of the solution to the boundary value problems.
\end{itemize}

In the rest of this section, I will examine a few examples. To make the interpretation easier, we will use the following deep theorem that relates the cohomology relative to the boundary \(H^k_{dR}(M,\de M)\) with the singular homology with real coefficients \(H_k(M;\R)\) (see \cite[Theorem~3.43]{HatcherAlgebraic_2000}).
\begin{thm}[Poincaré-Lefschetz duality]\label{thm: Poincaré-Lefscetz duality}
    Let \(M\) be a compact\footnote{In particular, it is a manifold of finite type, so cohomology and homology groups (with real coefficients) are finite dimensional vector spaces.} \(n\)-manifold with boundary \(\de M\). Then the map
    \begin{align*}
        H^k_{dR}(M,\de M) \times H^{n-k}_{dR}(M) &\to \R \\
        ([\omega], [\eta]) \mapsto \int_M \omega\wedge \eta
    \end{align*}
    is a well defined non degenerate pairing. In other words, by de Rham theorem\footnote{The de Rham theorem states that the de Rham cohomology is equivalent to the singular cohomology with real coefficients, i.e.,
    \[
        H^p_{dR}(M) \equiv H^p(M;\R) = H_p(M;\R)^*.
    \]
    See \cite[Chapter~18]{LeeSmooth_2013}.}
    \[
        H^k_{dR}(M,\de M) \equiv H_{n-k}(M;\R).
    \]
    canonically.
\end{thm}

\begin{ex}[Solid torus]
    Consider a solid torus 
    \[
        M = \{ (\cos \theta x, \sin \theta x, z) \in \R^3 \colon (x-2)^2+z^2 \le 1, \ \theta \in [0,2\pi]\} \cong \D^2 \times \Sp^1.
    \]
    Then \(M\) is homotopically equivalent to \(\Sp^1\), so \(H_\bullet(M) = H_\bullet(\Sp^1) = \R[t]/(t^2)\), and by Theorem~\ref{thm: Poincaré-Lefscetz duality}
    \begin{align*}
        H^0_{dR}(M,\de M) = 0,  & &H^1_{dR}(M, \de M)=0, \\
        H^2_{dR}(M,\de M) = \R, & &H^3_{dR}(M,\de M) = \R.
    \end{align*}
    The non triviality of \(H^2_{dR}(M,\de M)\) is caused by the presence of the non trivial loop \(\{0\} \times \Sp^1\), the ‘‘soul'' of the solid torus. Let us denote by \(\beta \in \Hcal^2_D(M)\) the harmonic representative of the non trivial class associated to it. 
    
    The electric field is always determined by \(E = \dif \phi\), where the scalar potential \(\phi \in \Omega^0_D(M)\) is the unique solution of the Poisson equation
    \[
        \triangle \phi = \rho,
    \]
    which always exists because \(\Hcal^0_D(M) \cong H^0_{dR}(M,\de M) = 0\).
    
    The magnetic field is \(B = \beta_B + \dif A\), where \(\beta_B \in \Hcal^2_D(M)\) is the harmonic representative of the class of \(B\) and \(A \in \Omega^1_D\) is the solution of
    \[
        \triangle A = J,
    \]
    which always exists because \(\Hcal^1_D(M) \cong H^1_{dR}(M,\de M) = 0\).
\end{ex}

\begin{ex}[Ball with spherical hole]
    Let 
    \[
        M= \{ x \in \R^3 \colon 1\le |x| \le 2\} \cong \Sp^2 \times [0,1].
    \]
    Then \(M\) is homotopically equivalent to \(\Sp^2\), so \(H_\bullet(\Sp^2;\R) = (\R[t]/(t^2), \deg(t)=2)\), and by Theorem~\ref{thm: Poincaré-Lefscetz duality}
    \begin{align*}
        H^0_{dR}(M,\de M) = 0,  & &H^1_{dR}(M, \de M)=\R, \\
        H^2_{dR}(M,\de M) = 0, & &H^3_{dR}(M,\de M) = \R.
    \end{align*}
    The non triviality of \(H^1_{dR}(M,\de M)\) is caused by the presence of the non trivial ‘‘middle sphere'' \(\Sp^2 \times \{1/2\}\).  

    The electric field is \(E = \alpha_E + \dif \phi\), where \(\alpha_E \in \Hcal^1_D(M)\) is the harmonic representative of the class of \(E\) and \(\phi \in \Omega^0_D\) is the solution of
    \[
        \triangle \phi = \rho,
    \]
    which always exists because \(\Hcal^0_D(M) \cong H^0_{dR}(M,\de M) = 0\).

    The magnetic field is \(B = \dif A\), where \(A \in \Omega^1_D\) is the solution of
    \[
        \triangle A = J,
    \]
    which exists if and only if the current density satisfies the compatibility condition
    \[
        \int_M \alpha \wedge \star J = 0,
    \]
    where \(\alpha \in \Hcal^1_D(M)\) is the harmonic representative of any non trivial class of \(H^1_{dR}(M,\de M)\), for instance the one associated to the ‘‘middle sphere''.
\end{ex}

\begin{ex}[Ball with toroidal hole]
    Let \(M = \B^3 \setminus N \subset \R^3\), where
    \[
    \begin{split}
        N &\coloneq \{ (\cos \theta x, \sin \theta x, z) \in \R^3 \colon (x-1/2)^2+z^2 < 1/16, \ \theta \in [0,2\pi]\} \\
        &\cong \D^2 \times \Sp^1.
    \end{split}
    \]
    A simple computation using the Mayer-Vietoris theorem (\(\B^3 = M \cup V\), with \(V \sim N\) a small neighborhood of \(N\)) shows that
    \begin{align*}
        H_0(M) = \R,  & &H_1(M)=\R, \\
        H_2(M) = \R, & &H_3(M) = 0.
    \end{align*}
    By Theorem~\ref{thm: Poincaré-Lefscetz duality},
    \begin{align*}
        H^0_{dR}(M,\de M) = 0,  & &H^1_{dR}(M, \de M)=\R, \\
        H^2_{dR}(M,\de M) = \R, & &H^3_{dR}(M,\de M) = \R.
    \end{align*}
    The non triviality of \(H^2_{dR}\) is caused by the presence of the loop
    \[
        \Gamma \coloneq \{(x,0,z) \in \R^3 \colon (x-1/2)^2 + z^2 = 1/8\} \cong \Sp^1,
    \]
    while the non triviality of \(H^1_{dR}(M)\) is caused by the presence of the ‘‘middle torus''
    \[
        T \coloneq \{ (\cos \theta x, \sin \theta x, z) \in \R^3 \colon (x,z) \in \Gamma, \ \theta \in [0,2\pi]\} \cong \Sp^1 \times \Sp^1.
    \]

    The electric field is \(E = \alpha_E + \dif \phi\), where where \(\alpha_E \in \Hcal^1_D(M)\) is the harmonic representative of the class of \(E\) and \(\phi \in \Omega^0_D\) is the solution of
    \[
        \triangle \phi = \rho,
    \]
    which always exists because \(\Hcal^0_D(M) \cong H^0_{dR}(M,\de M) = 0\).

    The magnetic field is \(B = \beta_B + \dif A\), where \(\beta_B \in \Hcal^2_D(M)\) is the harmonic representative of the class of \(B\) and \(A \in \Omega^1_D\) is the solution of
    \[
        \triangle A = J,
    \]
    which exists if and only if the current density satisfies the compatibility condition
    \[
        \int_M \alpha \wedge \star J = 0,
    \]
    where \(\alpha \in \Hcal^1_D(M)\) is the harmonic representative of any non trivial class of \(H^1_{dR}(M,\de M)\), for instance the one associated to the ‘‘middle torus'' \(T\).
\end{ex}

\printbibliography

\end{document}
